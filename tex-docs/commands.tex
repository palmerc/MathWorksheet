%----------------COMMANDS-------------------------------------------%%
\newcommand\blank{\underline{\hspace{2cm}}} % Gives a blank
\newcounter{prob} % A new counter for current problem number
\setcounter{prob}{1} % Start the counter at the value 1
\newcommand\itm{
\fbox{\textbf{\theprob}} \refstepcounter{prob}
} % Calls problem number
\newcommand{\problem}[1]{\makebox[0.5cm]{\itm}
  \begin{minipage}[t]{\textwidth-0.5cm} #1 \end{minipage}
} % An environment for a problem statement on or more lines
\newcommand{\pairofprobs}[2]{
  \begin{minipage}[t]{0.5\textwidth}\itm #1 \end{minipage}
  \begin{minipage}[t]{0.5\textwidth}\itm #2 \end{minipage}
} % Fits two problems on a line
\newcommand{\threeprobs}[3]{
\begin{minipage}[t]{0.31\textwidth}\itm #1 \end{minipage} \hfill
 \begin{minipage}[t]{0.31\textwidth}\itm #2 \end{minipage} \hfill
 \begin{minipage}[t]{0.31\textwidth}\itm #3 \end{minipage}
} % Fits three problems on a line
\newcounter{choice} % Counter for multiple choice problems
\setcounter{choice}{1} % Start the counter at the value 1
\newcommand\achoice{
(\alph{choice}) \stepcounter{choice}
} % Generates letter for multiple choice option
\newcommand{\answers}[5]{\vspace*{-7mm}
  \begin{tabular}{l@{\hspace{1mm}}p{0.9\textwidth}}
    \achoice & #1 \\ \achoice & #2 \\ \achoice & #3 \\
    \achoice & #4 \\ \achoice & #5 \end{tabular}
  \setcounter{choice}{1}
} % Makes multiple-choice options
%---------------------------------

% The commands below are for setting up arithmetic
% problems with the four basic operations. See examples
% in the CONTENT section

\newcommand\divi[2]{
$#1 \: \begin{array}{|l}
\hline #2
\end{array}$
}

\newcommand\mult[2]{
$\begin{array}{rr}
 & #1 \\
 \times & #2 \\ \hline
 \end{array}$}

\newcommand\addi[2]{
  $\begin{array}{rr}
   &  #1 \\
    + & #2 \\ \hline
  \end{array}$}

\newcommand\subt[2]{
  $\begin{array}{rr}
    & #1 \\
    - & #2 \\ \hline
  \end{array}$}
%-------------------------------------------------------------------%%
